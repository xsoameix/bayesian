% xelatex -interaction=nonstopmode -jobname=README note.tex
\documentclass{article}

\usepackage{fontspec}
\usepackage{xeCJK}
\usepackage{amsmath}
\usepackage[a4paper]{geometry}
\setCJKmainfont{WenQuanYi Micro Hei}
\XeTeXlinebreaklocale "zh"
\XeTeXlinebreakskip = 0pt plus 1pt
\title{筆記}
\author{Lien Chiang}
\date{}

\begin{document}
\maketitle

\section{離散分佈}

\bigskip
\subsection{Geometric Distribution 幾何分佈}

\bigskip
\subsubsection{密度函數}
\[ f(x)=q^{x-1}p \]

\subsubsection{累積函數}
\[ F(x)=\sum_{x=1}^\infty{q^{x-1}p}
       =p\frac{1-q^x}{1-q}
       =p\frac{1-q^x}{p}
       =1-q^x \]

\subsubsection{期望值}
\begin{align*}
E[X]
& =  \sum_{x=1}^\infty{xq^{x-1}p}
  = p\sum_{x=1}^\infty{xq^{x-1}}
  = p\sum_{x=1}^\infty{\frac{dq^x}{dq}} \\
& = p\frac{d}{dq}(q\frac{1}{1-q})
  = p\frac{q'(1-q)-q(1-q)'}{(1-q)^2} \\
& = p\frac{(1-q)+q}{p^2}
  = p\frac{1}{p^2}
  =  \frac{1}{p}
\end{align*}

\clearpage

\subsubsection{變異數}

\begin{align*}
Var(X)
& = E[(X-\mu)^2]
  = E[(X^2-2\mu X+\mu^2]
  = E[X^2]-2\mu E[X]+\mu^2 \\
& = E[X^2]-2E[X]^2+E[X]^2
  = E[X^2]-E[X]^2 \\
E[X^2] & =\sum_{x=1}^\infty{x^2 q^{x-1}p}
  = p\sum_{x=1}^\infty{x^2 q^{x-1}}
\intertext{由於我們無法計算出 \( x^2 q^{x-1} \) 的積分,式子的 \( x^2 q^{x-1} \) 無法代換成 \( \frac{d f(q)}{dq} \) 的形式。}
\intertext{藉由 \( x(x-1)q^{x-2} = \frac{d q^x}{dq} \),我們可以改算 \( E[X(X-1)] \)。}
E[X(X-1)]
& =   \sum_{x=1}^\infty{x(x-1)q^{x-1}p}
  = pq\sum_{x=1}^\infty{x(x-1)q^{x-2}}
  = pq\sum_{x=1}^\infty{\frac{d q^x}{dq^2}} \\
& = pq\frac{d}{dq^2}(\frac{1}{1-q})
  = pq(\frac{0(1-q)-1(-1)}{(1-q)^2})' \\
& = pq(\frac{1}{(1-q)^2})'
  = pq(\frac{0(1-q)^2-1\cdot2(1-q)(-1)}{(1-q)^4})
  = pq\frac{2p}{p^4}
  =   \frac{2q}{p^2} \\
E[X^2] & = E[X(X-1)]+E[X]
  = \frac{2q}{p^2}+\frac{1}{p}
  = \frac{2q+p}{p^2}
  = \frac{2(1-p)+p}{p^2}
  = \frac{2-p}{p^2} \\
Var(X) & = E[X^2]-E[X]^2
  = \frac{2-p}{p^2}-\frac{1}{p^2}
  = \frac{1-p}{p^2}
  = \frac{q}{p^2}
\end{align*}

\subsubsection{動差生成函數}
\begin{align*}
m_x(t)
& = E[e^{tX}]
  = \sum_{x=1}^\infty{e^{tx}q^{x-1}p}
  = p\sum_{x=1}^\infty{e^{tx}q^{x-1}} \\
\intertext{把 \( x-1 \) 提出來}
& = p\sum_{x=1}^\infty{e^{t(x-1)}e^t q^{x-1}}
  = pe^t\sum_{x=1}^\infty{(qe^t)^{x-1}}
  = pe^t\frac{1}{1-qe^t}
  = \frac{pe^t}{1-qe^t}
\end{align*}

\clearpage

\subsection{Binomial Distribution 二項分佈}

\bigskip
\subsubsection{密度函數}
\[ f(x)=\binom{n}{x}p^x q^{n-x} \]

\subsubsection{累積函數}
\[ F(x)=\sum_{k=0}^x{\binom{n}{k}p^k q^{n-k}} \]

\subsubsection{期望值}
\begin{align*}
\end{align*}

\subsubsection{變異數}
\begin{align*}
\end{align*}

\subsubsection{動差生成函數}
\begin{align*}
\end{align*}

\subsection{Negative Binomial Distribution 負二項分佈}

\bigskip
\subsubsection{密度函數}
\[ f(x)=\binom{x-1}{r-1}p^r q^{x-r} \]

\subsubsection{累積函數}
\[ F(x)=\sum_{k=r}^x{\binom{k-1}{r-1}p^r q^{k-r}}
       =\sum_{l=0}^x{\binom{l+r-1}{r-1}p^r q^l} \]

\subsubsection{期望值}
\begin{align*}
E[X]
& = \sum_{x=0}^\infty{x\binom{x+r-1}{r-1}p^r q^x}
  = \sum_{x=1}^\infty{x\binom{x+r-1}{r-1}p^r q^x}
  = \sum_{x=1}^\infty{\frac{(x+r-1)!}{(r-1)!\,(x-1)!}\,p^r q^x} \\
& = \sum_{y=0}^\infty{\frac{(y+r)!}{(r-1)!\,y!}\,p^r q^{y+1}}
  = \sum_{y=0}^\infty{\frac{(y+s-1)!}{(s-2)!\,y!}\,p^{s-1}q^{y+1}}
  = \frac{(s-1)q}{p}\sum_{y=0}^\infty{\frac{(y+s-1)!}{(s-1)!\,y!}\,p^s q^y} \\
& = \frac{rq}{p}\sum_{y=0}^\infty{\binom{y+s-1}{s-1}p^s q^y}
  = \frac{rq}{p}
\end{align*}

\subsubsection{變異數}
\begin{align*}
Var(X)
& = E[X^2]-E[X]^2 \\
E[X^2] & = \sum_{x=0}^\infty{x^2\binom{x+r-1}{r-1}p^r q^x}
  = \sum_{x=1}^\infty{x^2\binom{x+r-1}{r-1}p^r q^x}
  = \sum_{x=1}^\infty{x\frac{(x+r-1!)}{(r-1)!\,(x-1)!}p^r q^x} \\
& = \sum_{y=0}^\infty{(y+1)\frac{(y+r!)}{(r-1)!\,y!}\,p^r q^{y+1}}
  = \sum_{y=0}^\infty{(y+1)\frac{(y+s-1!)}{(s-2)!\,y!}\,p^{s-1} q^{y+1}} \\
& = \frac{(s-1)q}{p}\sum_{y=0}^\infty{(y+1)\frac{(y+s-1!)}{(s-1)!\,y!}\,p^s q^y}
  = \frac{rq}{p}\sum_{y=0}^\infty{(y+1)\frac{(y+s-1!)}{(s-1)!\,y!}\,p^s q^y} \\
& = \frac{rq}{p}\sum_{y=0}^\infty{(y+1)\binom{y+s-1}{s-1}p^s q^y} \\
& = \frac{rq}{p}\bigg(
  \sum_{y=0}^\infty{y\binom{y+s-1}{s-1}p^s q^y} +
  \sum_{y=0}^\infty{\binom{y+s-1}{s-1}p^s q^y}\bigg) \\
& = \frac{rq}{p}(\frac{sq}{p}+1)
  = \frac{rq}{p}(\frac{(r+1)q}{p}+1)
  = \frac{rq}{p}(\frac{rq}{p}+\frac{q}{p}+1) \\
Var(X) & = E[X^2]-E[X]^2
  = \frac{rq}{p}(\frac{rq}{p}+\frac{q}{p}+1) - \frac{rq}{p}\frac{rq}{p}
  = \frac{rq}{p}(\frac{q}{p}+1)
  = \frac{rq}{p}(\frac{q+p}{p})
  = \frac{rq}{p^2}
\end{align*}

\clearpage

\subsubsection{動差生成函數}
\begin{align}
\intertext{泰勒展開式}
f(x)
& = \frac{f^{(0)}(a)}{0!}(x-a)^0+
    \frac{f^{(1)}(a)}{1!}(x-a)^1+
    ...+
    \frac{f^{(k)}(a)}{k!}(x-a)^k+
    ...\,\bigg|_{a=0} \nonumber \\
& = \frac{f^{(0)}(0)}{0!}x^0+
    \frac{f^{(1)}(0)}{1!}x^1+
    ...+
    \frac{f^{(k)}(0)}{k!}x^k+
    ... \nonumber \\
\intertext{Let \( f(x)=(x+1)^r \)}
(x+1)^r & = \frac{(a+1)^r}{0!}x^0+
    \frac{r(a+1)^{r-1}}{1!}x^1+
    \frac{r(r-1)(a+1)^{r-2}}{2!}x^2+
    ... \nonumber \\
& + \frac{r!\,(a+1)^{r-k}}{k!\,(r-k)!}x^k+
    ...\,\bigg|_{a=0} \nonumber \\
& = \frac{1}{0!}x^0+
    \frac{r}{1!}x^1+
    \frac{r(r-1)}{2!}x^2+
    ...+
    \frac{r!}{k!\,(r-k)!}x^k+
    ... \nonumber \\
& = \binom{r}{0}x^0+
    \binom{r}{1}x^1+
    \binom{r}{2}x^2+
    ...+
    \binom{r}{k}x^k+
    ...
  = \sum_{k=0}^\infty{\binom{r}{k}x^k} \nonumber \\
(x+1)^{-r} & = \sum_{k=0}^\infty{\binom{-r}{k}x^k}
  = \sum_{k=0}^\infty{\frac{(-r)(-r-1)...(-r-k+1)}{k!}x^k} \nonumber \\
& = \sum_{k=0}^\infty{\frac{(-1)^k\,r(r+1)...(r+k-1)}{k!}x^k}
  = \sum_{k=0}^\infty{\frac{r(r+1)...(r+k-1)}{k!}(-x)^k} \nonumber \\
& = \sum_{k=0}^\infty{\frac{(r+k-1)!}{k!\,(r-1)!}(-x)^k}
  = \sum_{k=0}^\infty{\binom{k+r-1}{r-1}(-x)^k} \nonumber \\
(-x+1)^{-r} & = (1-x)^{-r}
  = \sum_{k=0}^\infty{\binom{k+r-1}{r-1}x^k} \\
m_t(x) & = E[e^{tx}] \nonumber \\
& = \sum_{x=0}^\infty{e^{tx}\binom{x+r-1}{r-1} p^r q^x}
  = \sum_{x=0}^\infty{\binom{x+r-1}{r-1} p^r (qe^t)^x}
  = p^r \sum_{x=0}^\infty{\binom{x+r-1}{r-1}(qe^t)^x} \nonumber \\
\intertext{代入 (1) 式}
& = p^r(1-qe^t)^{-r}
  = (\frac{p}{1-qe^t})^r \nonumber
\end{align}

\end{document}
