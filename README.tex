\documentclass{article}

\usepackage{fontspec}
\usepackage{xeCJK}
\usepackage{amsmath}
\setCJKmainfont{WenQuanYi Micro Hei}
\XeTeXlinebreaklocale "zh"
\XeTeXlinebreakskip = 0pt plus 1pt
\title{筆記}
\author{Lien Chiang}
\date{}

\begin{document}
\maketitle

\section{離散分佈}

\bigskip
\subsection{Geometric Distribution 幾何分佈}

\bigskip
\subsubsection{密度函數}
\[ f(x)=q^{x-1}p \]

\subsubsection{累積函數}
\[ F(x)=p\frac{1-q^x}{1-q}
       =p\frac{1-q^x}{p}=
        1-q^x \]

\subsubsection{期望值}
\begin{align*}
E[X]
& =  \sum_{x=1}^\infty{xq^{x-1}p}
  = p\sum_{x=1}^\infty{xq^{x-1}}
  = p\sum_{x=1}^\infty{\frac{dq^x}{dq}} \\
& = p\frac{d}{dq}(q\frac{1}{1-q})
  = p\frac{q'(1-q)-q(1-q)'}{(1-q)^2} \\
& = p\frac{(1-q)+q}{p^2}
  = p\frac{1}{p^2}
  =  \frac{1}{p}
\end{align*}

\clearpage

\subsubsection{變異數}

\bigskip
先算 \( Var(X) \)

\begin{align*}
Var(X)
& = E[(X-\mu)^2]
  = E[(X^2-2\mu X+\mu^2]
  = E[X^2]-2\mu E[X]+\mu^2 \\
& = E[X^2]-2E[X]^2+E[X]^2
  = E[X^2]-E[X]^2
\end{align*}

\bigskip
再算其中的 \( E[X^2] \)

\[ E[X^2]=\sum_{x=1}^\infty{x^2 q^{x-1}p}
         =p\sum_{x=1}^\infty{x^2 q^{x-1}} \]

\bigskip
不過由於我們無法計算出 \( x^2 q^{x-1} \) 的積分,式子的 \( x^2 q^{x-1} \) 無法代換成 \( \frac{d f(q)}{dq} \) 的形式。
相反的 \( x(x-1)q^{x-2} = \frac{d q^x}{dq} \) 卻成立,我們可以利用該式對 \( E[X] \) 做些手腳:算 \( E[X(X-1)] \) 作為替代。

\begin{align*}
E[X(X-1)]
& =   \sum_{x=1}^\infty{x(x-1)q^{x-1}p}
  = pq\sum_{x=1}^\infty{x(x-1)q^{x-2}}
  = pq\sum_{x=1}^\infty{\frac{d q^x}{dq^2}}
  = pq\frac{d}{dq^2}(\frac{1}{1-q}) \\
& = pq(\frac{0(1-q)-1(-1)}{(1-q)^2})'
  = pq(\frac{1}{(1-q)^2})'
  = pq(\frac{0(1-q)^2-1\cdot2(1-q)(-1)}{(1-q)^4}) \\
& = pq\frac{2p}{p^4}
  =   \frac{2q}{p^2} \\
\bigskip \\
E[X^2] & = E[X(X-1)]+E[X] \\
& = \frac{2q}{p^2}+\frac{1}{p}
  = \frac{2q+p}{p^2}
  = \frac{2(1-p)+p}{p^2}
  = \frac{2-p}{p^2} \\
\bigskip \\
Var(X) & = E[X^2]-E[X]^2 \\
& = \frac{2-p}{p^2}-\frac{1}{p^2}
  = \frac{1-p}{p^2}
  = \frac{q}{p^2}
\end{align*}

\subsubsection{動差生成函數}

\bigskip
技巧:把 \( x-1 \) 提出來

\begin{align*}
m_x(t)
& = E[e^{tX}]
  = \sum_{x=1}^\infty{e^{tx}q^{x-1}p}
  = p\sum_{x=1}^\infty{e^{tx}q^{x-1}}
  = p\sum_{x=1}^\infty{e^{t(x-1)}e^t q^{x-1}}
  = pe^t\sum_{x=1}^\infty{(qe^t)^{x-1}} \\
& = pe^t\frac{1}{1-qe^t}
  = \frac{pe^t}{1-qe^t}
\end{align*}

\clearpage

\end{document}
